\begin{abstract}
\setlength{\parskip}{0.5cm}

El fin de este trabajo de fin de grado es recrear una antigua página web que explotaba una base de datos orientada a la pigmentación. La aplicación antigua hecha en forma de web ha sido trasladada a una aplicación móvil capaz de ejecutarse en cualquier dispositivo con Android que hay en la actualidad. 

Primero se ha tenido que recuperar la información existente desde un dispositivo de almacenamiento antiguo, como un CD. Una vez analizada la poca documentación actual y revisada la estructura de la base de datos, se ha procedido a trasladarla a una base de datos más actual, en este caso se ha decidido usar una base de datos SQLite. Una vez hemos diseñado e implementado la base de datos de la manera correcta, se ha procedido a desarrollar la aplicación móvil. En este caso se ha decidido usar Java como lenguaje de programación por la familiaridad del lenguaje a la hora de programar. 

Cabe destacar que la metodología que se ha implementado para llevar a cabo este trabajo refleja un poco la de Kanban, ya que el proyecto se ha apoyado en una herramienta de gestión de proyectos de código abierto y libre como es Taiga.io, donde se puede ver todo el desarrollo, fechas y tiempo del proyecto, junto con los problemas que se han ido teniendo y solucionando. 

Como consecuencia de la metodología utilizada, se han mezclado las fases de desarrollo con las fases de diseño, además de que no todas las partes de diseño se ven cubiertas de manera completa. 
\end{abstract}