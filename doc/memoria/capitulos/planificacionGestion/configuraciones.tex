La gestión de la configuración en este proyecto no va a ser su punto fuerte. El objetivo de la gestión de la configuración es mantener la integridad de los artefactos que se obtienen en cada uno de los proyectos que integran el área garantizando que no se realizan cambios no controlados y que todos los participantes del proyecto disponen de la versión adecuada de los productos y artefactos que manejan. 

Así, entre los elementos de configuración software, se encuentran no únicamente ejecutables y código fuente, sino también los modelos de datos, modelos de procesos, especificaciones de requisitos, pruebas, etc. La gestión de configuración es una actividad continúa ya que se realiza durante todas las actividades asociadas al desarrollo de un sistema, y continúa registrando los cambios hasta que éste deja de utilizarse. Es una actividad de garantía de calidad que se aplica en todas las fases del proceso de ingeniería del software.

En nuestro caos no se va a usar un software específico de gestión de configuraciones. Además la gestión de configuraciones no es solo una base de datos donde se mantiene la información. Es un proceso continuo que avanza en paralelo al desarrollo del proyecto. Un ejemplo de una buena gestión de configuraciones habla de la necesidad de realizar auditorías, pero en un proyecto tan pequeño como el que vamos a desarrollar no tiene sentido.

En nuestro caso para mantener un mínimo control sobre las configuraciones y las versiones, no solo del código fuente y y de los ejecutables, sino también de los diagramas y de los documentos vamos a usar el mismo SCV que para el código fuente. Git. 