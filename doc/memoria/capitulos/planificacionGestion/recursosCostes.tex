\subsubsection{Recursos y tipos}

En lo que a estimación de recursos y costes se refiere, este proyecto es modesto, es decir que no va a disponer ni depender de caros y grandes recursos. Podemos tener varios tipos de recursos, al ser un proyecto pequeño, se van a dividir en 3 grandes categorías:
\begin{itemize}
\item \textbf{Personales}: a nivel de personal solo se involucran a 3 personas reales. Una es el director del TFG, a la cual no se le puede contar casi como recurso, ya que es mas la persona que desbloquea los principales problemas que nos podemos ir encontrando conforme vamos realizando el proyecto. Otra de las personas es el profesor asociado al cual va destinada la aplicación, esta persona si que es más importante porque hay otros recursos que han dependido de ella. Otro de los recursos personales es el propio autor del documento, el cual ha tenido que poner en práctica conocimientos de diseño de interfaces, software arquitectura, redacción... para lograr este documento. Luego haremos los cálculos de coste, de momento simplemente estamos planteando los recursos.

\item \textbf{Físicos}: a nivel de recursos físicos solo vamos a precisar de un ordenador para realizar el proyecto y la documentación del mismo, el ordenador no tiene que ser demasiado potente ya que solo tiene que cargar un IDE para desarrollar en aplicaciones Android. También podemos tener en cuenta los periféricos asociados y finalmente podemos añadir el móvil de pruebas en el que se van a realizar las principales pruebas físicas durante el desarrollo de la aplicación. 

\item \textbf{Datos}: voy a considerar como recursos los datos disponibles. Cuando empezó el trabajo la base de datos venía en forma de datos almacenados en un CD, pero en el futuro se ha visto que había por ejemplo falta de ese tipo de recursos.

\item \textbf{Otros}: podemos entrar a considerar recursos que para este proyecto no tiene casi sentido tener en cuenta. Dichos recursos pueden ser la electricidad gastada durante se desarrollaba el proyecto, o simplemente la estancia física donde se realizaba la acción. Repito que en este caso no tiene sentido tenerlos en cuenta pero si que me gustaría que quedase claro que en proyecto más grandes con muchas más personas, este tipo de recursos hay que contabilizarlos, controlarlos y tenerlos muy en cuenta (el precio de alquiler de una oficina puede disparar el precio de desarrollo por ejemplo).
\end{itemize}

\subsubsection{Estimación de los costes}

Intentar dar una estimación lo más adecuada posible de el coste de desarrollo total de este proyecto.
\begin{enumerate}
\item \textbf{Costes físicos}: estimar los costes de los recursos físicos anteriormente nombrados mediante la siguiente suma: 
$$ C_{fisico} = 550 + 20 + 15 + 150 = 735 EUR$$
El desglose sería el siguiente: 
\begin{enumerate}
\item \textbf{Ordenador}: por unos 550 EUR se pueden encontrar ordenadores con las prestaciones que nosotros buscamos. Mas o menos tienen que tener 8GB de RAM y con un procesador del tipo Intel Core i3 o i5 es suficiente. Con un almacenamiento de 500GB de disco HDD (no se precisan tantos) se puede desarrollar perfectamente este trabajo en condiciones buenas. 
\item \textbf{Ratón}: 20 EUR es el precio de mercado medio de un ratón inalámbrico para usarlo con el ordenador.
\item \textbf{Teclado}: 20 EUR es el precio de mercado medio de un teclado externo para usarlo con el ordenador.
\item \textbf{Móvil}: 150 EUR es el precio medio de un móvil con prestaciones normales, es decir entre 1GB y 2GB de RAM, almacenamiento ROM de entre ( y 16 GB y un procesador, junto con pantalla y cámaras de prestaciones medias. 
\end{enumerate}

\item \textbf{Costes personales}: para los costes personales primero tenemos que tener una buena estimación temporal, que en nuestro caso es: X días. Para ver cuanto cuesta pagar a una persona y que desarrollo dicho proyecto vamos a suponer que la hora a la que ese trabajador ha desarrollado el proyecto ha sido pagada a: 5 EUR, que dedicando una jornada laboral completa suponen unos beneficios de 40 EUR/dia y 1200 EUR/mes (no vamos a contabilizar los impuesto para facilitar las operaciones). Como nuestro trabajador va a dedicar unos X días, el coste de esa persona es aproximadamente: 
$$ C_{personal} = X dias \cdot 40 EUR/dia = XXXX EUR$$

\item \textbf{Costes totales}: los costes totales son los siguientes: 
$$ C_{total} = C_{fisico} + C_{personal} = 735 + XXXX EUR$$
\end{enumerate}

\textcolor{red}{acabar los calculo del coste, dependen del timpo}
