Los proyectos software, como su nombre bien lo indica son proyectos, aunque a lo largo de la historia esta ultima parte se olvida a menudo. Desde que la informática comenzó en sus días, las empresas se embarcaban en grandes macro proyectos de cientos de miles de euros, pero la realidad es que muchos de ellos fracasaban y la principal de las razones era la falta de una adecuada planificación. Por lo menos con el tiempo, tanto las empresas como los profesionales y expertos en planificación han conseguido que esto no pase tanto, es decir, hay una tendencia alcista en lo que a éxitos en los proyectos de software se refiere, junto con un descenso significativo de los proyectos que al final no se acaban por unas u otras razones. 

La planificación es necesaria en todos los niveles, no hay ningún proyecto grande ni pequeño que se pueda empezar sin una planificación mínima. Incluso el proyecto más pequeño parte de una idea inicial que una persona tiene estructurada en la cabeza, lo cual es planificación. 

Cuando intentamos abordar proyectos de software que involucran a cientos de personas, recursos, cantidades grandes de dinero, se espera que ese proyecto salga adelante, pero la verdad es que ese proyecto solo saldrá adelante si se ha planificado y ejecutado de acuerdo a dicha planificación.

Uno de los principales problemas de la falta de planificación en los proyectos de software es la invisibilidad, por lo general planificar la construcción de un puente o el propio proyecto es mucho más vísale que la construcción de un artefacto de software. Otra de las principales características que dificultan la planificación es la complejidad. No es lo mismo intentar coordinar a 2 personas para que levanten un muro, que desarrollar una aplicación multiplataforma, están objetivamente demostrado que los proyectos de software son más complejos que los tradicionales por unidad de coste.

Estas y otras características son las que al final llevan a tener una planificación de proyectos pobre o deficiente, lo cual puede desembocar en la no realización del proyecto, o en la realización del mismo pero son sobrecostes o funcionalidad que no se han llegado a desarrollar. 

