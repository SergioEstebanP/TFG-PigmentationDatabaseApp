La gestión de riesgos es un tema complicado, que intentaré abordar según la mayor brevedad y facilidad posible. Aunque la planificación de un proyecto se basa en describir, planificar, asignar y revisar las tareas y recursos conocidos, existen muchos recursos desconocidos o difíciles de establecer que pueden incidir directamente en la consecución de los resultados en el tiempo previsto. Un riesgo implica dos factores claves:
\begin{itemize}
\item \textbf{Incertidumbre}: probabilidad de que dicho riesgo se manifieste con el paso del tiempo.
\item \textbf{Pérdida}: si el riesgo se convierte en una realidad, ocurrirán consecuencias o pérdidas no deseadas.
\end{itemize}

EL proceso de tratamiento de los riesgos tiene unas fases bien claras y definidas. Muchas empresas optan por la negación de los riesgos y la corrección de los mismos junto con los problemas ocasionados mientras estos van apareciendo. Este comportamiento no es mas que el reflejo de una pobre gestión y planificación ya que los riesgos son potenciales problemas que pueden surgir durante el desarrollo y si no se tienen en cuenta, pueden dar lugar a la no consecución de los objetivos del proyecto o incluso a la no finalización del mismo.

Vamos a dar unas pequeñas clasificaciones de posibles riesgos y luego presentaremos un análisis de riesgos inicial. Estamos por lo tanto en la \textbf{fase de análisis de riesgos}: 
\begin{itemize}
\item \textit{Directos}: depende enteramente de tu proyecto.
\item \textit{Indirectos}: tu proyecto no puede hacer nada, están fuera del alcance. 
\end{itemize}

Otro tipo de clasificación que podemos dar:
\begin{itemize}
\item \textit{De proyecto}: amenazan el plan del proyecto, aumentan el tiempo y el coste generalmente.
\item \textit{Técnicos}: potenciales riesgos de implementación, arquitectura, diseño ...
\item \textit{De negocio}: no entender bien los requisitos, no es de calidad, no es intuitivo. 
\end{itemize}

Hemos hablado de que estamos en la fase de análisis de riesgos, una vez que hemos clasificado los riesgos deberíamos de estimar la probabilidad de que estos sucedan y el impacto al proyecto en caos de que se manifiesten. Al terminar esta fase deberíamos de pensar en posibles soluciones y alternativas para los problemas que se pueden ocasionar si dicho riesgo se manifiesta en el futuro. En conclusión la lista de riesgos, junto con la probabilidad, el impacto y la posible solución la podemos revisar a continuación. Además dicha lista está ordenada por prioridad:

\begin{itemize}
\item \textbf{Pérdida repentina de datos del PC}: 10\% de que ocurra - 0.9 de impacto - Posible solución: usar tecnologías de almacenamiento en la nube junto con la replicación de los datos en varios soportes de almacenamiento. 
\item \textbf{Mala gestión del tiempo}: 40\% de que ocurra - 0.95 de impacto - Posible solución: posponer la presentación de dicho trabajo.
\item \textbf{Diseño poco intuitivo de la interfaz}: 10\% de que ocurra - 0.5 de impacto - Posible solución: organizar reuniones periódicas con el cliente para prevenir la aparición del riesgo.
\item \textbf{Mal diseño de la base de datos}: 10\% de que ocurra - 0.2 de impacto - Posible solución: para prevenir podemos diseñar la base de datos de las maneras aprendidas en la carrera, aunque a veces no sean los mejores diseño nos podemos aferrar a buenos diseños de manera objetiva si hemos cumplido ciertas reglas. 
\item \textbf{Falta de datos}: 50\% de que ocurra - 0.8 de impacto - Posible solución: hablar con los profesores encargados para buscar o generar dichos datos. (manifestado)
\item \textbf{Explotación pobre de la base de datos}: 10\% de que ocurra - 0.2 de impacto - Posible solución: organizar charlas con el cliente para que presente todas las ideas posibles. 
\item \textbf{Bloqueo a nivel técnico por el lenguaje de programación}: 5\% de que ocurra - 0.1 de impacto - Posible solución: usar otro lenguaje de programación o mejorar las habilidades en el que se está usando.
\item \textbf{Mala arquitectura de la aplicación}: 15\% de que ocurra - 0.5 de impacto - Posible solución: aplicar patrones bien definidos y estudiados por personalidades importantes dentro del mundo de la arquitectura de software. Esta comprobado de manera objetiva que son buenas soluciones a problemas que ya se han planteado con anterioridad. 
\item \textbf{Bajo rendimiento}: 5\% de que ocurra - 0.3 de impacto - Posible solución: optimizar las búsquedas en la base de datos, estructuras de datos de la aplicación ...
\item \textbf{Baja calidad del producto}: 20\% de que ocurra - 0.1 de impacto - Posible solución: someter a la aplicación a un sistema de integración continua donde se prueba una cierta y mínima calidad, tanto a nivel de código como de funcionamiento de software. 
\end{itemize}

\textcolor{red}{aun hay que ordenar los riesgos según una prioridad que tenemos que establecer, pensar mas riesgos}