A continuación destacaremos algunas de las posibles aplicaciones que tiene este proyecto, incluyendo las posibles utilidades que supondrá su culminación para diversos colectivos. 

La utilización de determinados pigmentos y aglutinantes, así como sus mezclas, es en muchas ocasiones característico de la obra de un autor, así como de la época, zona de influencia y desarrollo del arte en un periodo histórico concreto. De esta forma, los estudios realizados a través de las técnicas instrumentales descritas anteriormente pueden ser aplicados en tareas de identificación, datación y autentificación de obras de arte. Los expertos encargados de realizar estas tareas pueden beneficiarse de la creación de la base de datos, ya que podrán recurrir a ella para realizar contrastes entre los registros obtenidos en el laboratorio y aquellos recogidos en la base.

Los profesionales de empresas de restauración necesitan en ocasiones reintegrar partes dañadas siguiendo las técnicas originales. Requieren por lo tanto conocer los orígenes de los pigmentos empleados con el fin de acometer las obras de restauración en las mejores condiciones. Estos expertos pueden beneficiarse de la creación de la base de datos, puesto que podrán acceder a diverso tipo de información obtenida a través de potentes herramientas analíticas de una forma fácil y con una disponibilidad total una vez registrados como usuarios autorizados.

Generar una aplicación móvil disponible en un dispositivo que podemos llevar a cualquier lugar es una importante innovación por el hecho de que esta herramienta puede ser utilizada de manera directa e inmediata por las empresas y por los expertos de instituciones y organismos dedicados a la restauración y conservación del patrimonio histórico-artístico, que muchas veces deben de decidir en un corto espacio de tiempo el tratamiento más adecuado para restaurar una determinada obra. Cada uno de los restauradores que tenemos en el territorio puede tener instalada dicha aplicación y visualizar métricas y datos mientras realiza trabajos de campo, sin necesidad de acceso a Internet, de una manera rápida y sencilla. La disponibilidad de la información es crucial en el desarrollo de este proyecto. 

De la misma forma, la aplicación móvil desarrollada permitirá poner en contacto a diversos colectivos de profesionales que realizan sus investigaciones en todo el mundo, así como obtener información complementaria a estas como bibliografía, ya sea en forma de libro o medio digital.

