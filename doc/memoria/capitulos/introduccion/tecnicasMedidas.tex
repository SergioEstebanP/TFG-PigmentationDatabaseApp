En esta parte vamos a hablar de las diferentes técnicas experimentales que han sido usadas para obtener los pigmentos, junto con algunas de las medidas que vamos a presentar dentro de la aplicación. Esto no es del ámbito informático pero tiene una cierta importancia a la hora de entender las gráficas que más adelante se mostrarán a los diferentes usuarios. 

Las principales técnicas y métodos de análisis de los que se han dispuesto para la elaboración de las bases de datos son los siguientes:
\begin{itemize}
\item \textbf{Análisis micro colorimétrico - Determinación de coordenadas tricrómicas}: las coordenadas tricrómicas permiten situar al pigmento dentro de un mapa cromático sin ambigüedad alguna, posibilitando el estudio del grado de evolución cromófora del pigmento en función de su antigüedad, composición mineralógica, procedencia y posible evolución temporal y espacial, e interacción con otros pigmentos.
\item \textbf{Difracción de rayos-x}: el análisis del difractograma de rayos X, permiten obtener con rigor las fases mineralógicas presentes en el material pigmentante. Los minerales accesorios a veces son de relevante importancia dado que posibilitan determinar áreas de procedencia comunes y, por tanto, determinar si un pigmento concreto fue usado de forma generalizada en diversas obras de arte o estuvo restringido a obras y autores determinados.
\item \textbf{Espectroscopía infrarroja por transformada de Fourier con transmisión atenuada de reflectancia}: la técnica FTIR-ATR (Infrarrojo por Transformada de Fourier con Reflectancia Total Atenuada) permite identificar las fases iónicas y moleculares presentes y, a través de los parámetros espectrales característicos, se obtiene información suplementaria sobre el estado del pigmento y su proceso evolutivo temporal.
\item \textbf{Especstroscopía Raman}: la espectroscopía micro-Raman es una técnica que proporciona información sobre la estructura dinámico vibracional de las especies químicas que constituyen el pigmento, complementándose con la espectroscopía FTIR ATR. Los espectros dan una información complementaria a la estructura estática obtenida mediante DRX. Además, la espectroscopía micro-Raman permite la detección de posibles cambios físico-químicos inducidos en el material por la acción de la radiación láser, con la ventaja de carácter de inocuidad y su alta resolución espacial 1$\mu$.
\end{itemize}
\textcolor{red}{decir que esto esta sacado de la anterior memoria}