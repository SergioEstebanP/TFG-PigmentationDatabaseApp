Este proyecto consiste en la creación de una aplicación móvil que explote de una manera adecuada una base de datos ya existente sobre pigmentos naturales y sintéticos. Dicha base de datos posee información tanto de las propiedades colorimétricas como estructurales de cada pigmento, dichos datos han sido recogidos mediante la aplicación de técnicas experimentales en laboratorios con los instrumentos adecuados. Más adelante hablaremos brevemente.

Esta aplicación tiene que ofrecer al usuario la capacidad de obtener de manera sencilla e intuitiva información compleja acerca de los diferentes pigmentos. Además de cada uno de ellos vamos a hablar también acerca de sus diferentes usos y de las técnicas por las que se han obtenido dichas medidas, y quizá en algunos de ellos (en los que al científico director le parezca mas relevante) se podrán añadir notas curiosas, como por ejemplo de la obtención de dicho pigmento o de usos pintorescos del mismo a lo largo de la historia. 

La base de datos, como hemos dicho ya existe, la idea y objeto principal de este proyecto es el desarrollo de una aplicación capaz de funcionar en un porcentaje elevado de los dispositivos móviles actuales sino en todos. 
