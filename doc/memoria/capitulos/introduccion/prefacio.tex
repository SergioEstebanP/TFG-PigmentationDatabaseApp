Durante estos últimos años debido a la situación económica y política de España, el sector de la conservación y la restauración del patrimonio cultural se ha dejado un poco de lado, aunque sigue a la cabeza de tecnología para restauración. 

Al mismo tiempo, se trata de un campo de trabajo que demanda amplia información, especialmente en lo relativo a la utilización de nuevas tecnologías, tanto de tratamiento como de estudio, para evitar intervenciones desafortunadas que han hecho peligrar el estado de conservación de muchas obras de nuestro rico patrimonio. A día de hoy contamos con tecnología puntera en Europa como pueden ser los escáneres en 3D de zonas muy amplias, estos escáneres nos ofrecen imágenes de muy alta resolución que los historiadores y profesionales de la materia precian enormemente para evitar deteriora las obras reales.

Por citar algunas de las otras tecnologías que podemos encontrar en el ámbito de la restauración pueden ser: tomografía, microscopía óptica, envejecimiento artificial (cada vez mas desarrollado con los potentes ordenadores con los que contamos), espectrofotometría infrarroja por transformación de Fourier son algunos de los métodos que podemos aplicar cuando se trata de restaurar nuestro patrimonio. 

Este proyecto trata de facilitar tanto la búsqueda como la utilización de información relacionada con el sector de la restauración y conservación del patrimonio, para que profesionales dedicados a este campo puedan encontrar un punto de referencia práctico y de fácil acceso que les ayude en sus investigaciones y proyectos y que esté al alcance de todas las personas. 

