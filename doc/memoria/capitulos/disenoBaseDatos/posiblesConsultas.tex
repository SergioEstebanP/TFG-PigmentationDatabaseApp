Vamos a analizar de manera breve las diferentes consultas que podemos realizar contra la base de datos que tenemos que diseñar. En nuestro caso, las consultas quedan determinadas por las diferentes interfaces que vamos a tener en nuestra aplicación. En este caso la principal información que vamos a tener que extraer de la base de datos es la siguiente:

\begin{itemize}
    \item Información acerca de cada uno de los pigmentos: en nuestra aplicación tenemos una vista en la que se muestra una lista con todos los pigmentos junto con la información más básica de cada uno de ellos, como puede ser el nombre, el color y el elemento químico principal por el que están compuestos. Esta información la podemos extraer directamente de la relación Pigmentos, donde tenemos toda la información básica de todos ellos. 
    \item Coordenadas colorimétricas: cada uno de los pigmentos consta de una serie de coordenadas que determinan el color y la luminosidad del mismo. Estas coordenadas serán almacenadas en una relación diferente porque la considero una característica un poco más avanzada que por ejemplo el nombre del pigmento. Cada uno de los pigmentos tendrá una relación directa con sus coordenadas colirimétricas. Esta relación será posible gracias al identificador único del pigmento que será utilizado como clave primaria en todas las relaciones. 
    \item Difractograma de rayos X: cada uno de los pigmentos tiene información relacionada con su generación del difractograma de rayos X característico. Por ello esta información se ha almacenado en una tabla diferente y el método de acceso será equivalente al mencionado en la relación anterior. Tenemos que ser capaces de extraer una gráfica para poder mostrar el espectro de rayos X de cada uno de los elementos que tenemos en la base de datos. 
    \item Infrarojos: otra de las características representativa de cada uno de los pigmentos es su \textcolor{red}{mirar en la documentación las definiciones exactas de cada una de las cosas que estoy poniendo} espectro de infrarojos, básicamente va a estar compuesto por dos valores a partir de los cuales se podrán generar una serie de gráficas. Tenemos que ser capaces de extraer las gráficas correspondientes al espectro de infrarojos para cada uno de los pigmentos.
    \item Espectro de Raman: cada uno de los pigmentos tiene asociado un espectro de Raman característico. Este espectro se puede generar gracias a dos valores, ya que es una gráfica simple, en la que luego tenemos que dar los puntos más representativos. Tenemos que ser capaces de extraer los 3 picos de Raman más importantes junto con las gráficas correspondientes. 
\end{itemize}