El principal objetivo es facilitar tanto la búsqueda como la utilización de la información por parte de los posibles usuarios. El formato de presentación informatizado permite la actualización progresiva de la información presentada conforme se vaya ampliando la investigación así como una mayor interactividad por parte del usuario. Para conseguir este objetivo, la información contenida en la base de datos se presentará de forma sencilla de modo que la navegación y las consultas pueden ser llevadas a cabo de manera intuitiva y sin que sean precisos conocimientos previos.

Otro de los objetivos es facilitar la colaboración entre diversos grupos de investigación españoles y extranjeros que realizan estudios en el campo de la caracterización de pigmentos, así como facilitar otro tipo de información (bibliográfica) que pueda ser utilizada por las empresas y por los expertos de instituciones y organismos dedicados a la conservación del patrimonio histórico-artístico.

Además de los objetivos presentados anteriormente, podemos encontrar los siguientes beneficios derivados de la creación de dicha aplicación: 

\begin{itemize}
    \item La utilización de determinados pigmentos y aglutinantes, así como sus mezclas, es en muchas ocasiones característico de la obra de un autor, así como de la época, zona de influencia y desarrollo del arte en un periodo histórico concreto. De esta forma, los estudios realizados a través de las técnicas instrumentales descritas anteriormente pueden ser aplicados en tareas de identificación, datación y autenticación de obras de arte. Los expertos encargados de realizar estas tareas pueden beneficiarse de la creación de la base de datos, ya que podrán recurrir a ella para realizar contrastes entre los registros obtenidos en el laboratorio y aquellos recogidos en la base.
    \item Los profesionales de empresas de restauración necesitan en ocasiones reintegrar partes dañadas siguiendo las técnicas originales. Requieren por lo tanto conocer los orígenes de los pigmentos empleados con el fin de acometer las obras de restauración en las mejores condiciones. Estos expertos pueden beneficiarse de la creación de la base de datos, puesto que podrán acceder a diverso tipo de información obtenida a través de potentes herramientas analíticas de una forma fácil y con una disponibilidad total una vez registrados como usuarios autorizados.
\end{itemize}