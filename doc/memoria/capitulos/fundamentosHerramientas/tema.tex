A lo largo del proyecto se va a tratar de producir una aplicación móvil que sea potable y ejecutable en deferentes dispositivos para el acceso y explotación de una base de datos con información de diferentes pigmentos. La información es básicamente a cerca de las propiedades físico químicas de los diferentes materiales. El generar una aplicación disponible en un teléfono móvil al alcance de cualquiera permitirá tanto a los profesionales de empresas e instituciones dedicados a la conservación y restauración del patrimonio histórico-cultural como a estudiantes a tener disponible en cualquier momento la información necesaria acerca de los diferentes pigmentos utilizados en las restauraciones o encontrados en el lugar de explotación. 

Este proyecto consiste en la creación de una aplicación móvil para la explotación de una base de datos informatizada sobre pigmentos naturales y sintéticos que incluya información sobre sus propiedades colorimétricas y estructurales proveniente de la aplicación de diversas técnicas instrumentales. 

Toda esta información técnica será accesible a través de dicha aplicación móvil que permitirá realizar diversas consultas sobre la base de datos a usuarios previamente autorizados. También se incluirán datos acerca del empleo de cada pigmento a lo largo de la historia e información complementaria sobre libros, artículos y grupos de investigación relacionados. 